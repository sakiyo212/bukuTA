\chapter{PENDAHULUAN}
\label{chap:pendahuluan}

% Ubah bagian-bagian berikut dengan isi dari pendahuluan

Penelitian ini dilatarbelakangi oleh beberapa kondisi yang menjadi acuan. Terdapat beberapa permasalahan yang terkait dengan acuan tersebut, batasan dari permasalahan tersebut, serta terdapat tujuan dan manfaat yang akan dibahas dan dijawab sebagai hasil dari penelitian ini.

\section{Latar Belakang}
\label{sec:latarbelakang}

Pada era teknologi digital saat ini, penggunaan kecerdasan buatan sangatlah erat pada aspek kehidupan manusia. Mulai dari membantu produktifitas seperti: rekomendasi konten pada social media, asisten virtual, filter spam; meningkatkan efisiensi, seperti: system transportasi cerdas dan penjadwalan otomatis; hiburan, dan dalam sektor penelitian dan pengembangan dalam sains dan teknologi.

Retinopati diabetik adalah komplikasi mikrovaskular diabetes melitus (DM) yang disebabkan oleh kerusakan pembuluh darah di retina. Penyakit ini dapat menyebabkan penurunan penglihatan, bahkan kebutaan \parencite{Yusran2022}. Menurut Organisasi Kesehatan Dunia (WHO), sekitar 9,3 juta orang di dunia menderita kebutaan akibat diabetic retinopathy. Jumlah ini diperkirakan akan meningkat menjadi 12,6 juta pada tahun 2040.

Hal ini membuat diagnosis dini retinopati diabetik sangat penting untuk mencegah progresi penyakit dan mengurangi risiko komplikasi serius. Penggunaan teknologi dalam dunia medis, terutama di bidang pemrosesan citra medis, telah menjadi bagian integral dari upaya untuk meningkatkan deteksi dini retinopati diabetik. Salah satu metode yang dipahami dengan baik dan memiliki banyak alat yang dikembangkan untuk analisis lebih dalam adalah Residual Neural Network (ResNet).

ResNet adalah jaringan \emph{neural network} yang dirancang untuk mengatasi masalah penurunan kinerja pada jaringan saraf yang lebih dalam. Mekanisme residual memungkinkan ResNet untuk mengoptimalkan pembelajaran jaringan pada data yang kompleks, seperti gambar medis. Tujuan dari penelitian ini adalah untuk menganalisis retinopati diabetik dengan mengimplementasikan Konvolusi Jaringan Saraf Tiruan.

\section{Rumusan Masalah}
\label{sec:permasalahan}

Berdasarkan hal yang telah dipaparkan di latar belakang, didapatkan rumusan masalah sebagai berikut:
\begin{enumerate}
	\item Bagaimana model ResNet dapat dilatih untuk secara akurat mendeteksi keberadaan retinopati diabetik pada citra fundus mata?
	\item Apakah model ResNet dapat secara efektif mengklasifikasikan tingkat tingkat resiko retinopati diabetik (sehat, non-proliferatif, proliferatif) berdasarkan citra fundus mata?
	\item Seberapa signifikan perbedaan kinerja antara model ResNet yang diuji dalam deteksi retinopati diabetik?
	%Faktor-faktor apa yang mempengaruhi kinerja model ResNet dalam analisis retinopati diabetik pada citra fundus mata?
\end{enumerate}

\section{Tujuan}
\label{sec:Tujuan}

Tujuan dari penelitian ini adalah:
\begin{enumerate}
	\item Mengetahui kemampuan ResNet dalam mengidentifikasi keberadaan diabetic retinopathy pada citra fundus mata.
	\item Mengkaji performa variasi ResNet dalam mengklasifikasikan tingkat keparahan retinopati diabetik menjadi sehat, sedang, dan berat.
	\item Mengukur dan membandingkan perbedaan performa dari berbagai model ResNet untuk mengetahui apakah ada perbedaan yang signifikan.
	%Menganalisis cara model deep learning mendeteksi dan mengklasifikasikan retinopati diabetik. Analisis ini akan fokus pada faktor-faktor yang mempengaruhi kinerja ResNet dalam menganalisis citra fundus mata untuk membuat keputusan.
\end{enumerate}

\section{Batasan Masalah}
\label{sec:batasanmasalah}

Untuk memperfokus permasalahan yang diangkat maka diberikan batasan sebagai berikut:

Berdasarkan rumusan masalah yang telah dijelaskan sebelumnya, maka penelitian ini memiliki batasan masalah sebagai berikut:
\begin{enumerate}
	\item Pembatasan penyakit yaitu Diabetic Retinopathy.
	\item Pengelompokan berdasarkan tiga tingkatan: Sehat, non-proliferatik, dan proliferatik
	\item Dataset yang digunakan berasal dari Diabetic Retinopathy Analisis Grand Challenge.
	\item Citra fundus yang dipakai merupakan citra fundus OCT-A dalam bentuk hitam-putih.
\end{enumerate}

\section{Manfaat}
\label{sec:Manfaat}

\begin{enumerate}
	\item Meningkatkan akurasi diagnosis retinopati diabetik
	\item Mempercepat proses diagnosis retinopati diabetik
	\item Meningkatkan ketersediaan layanan retinopati diabetik
\end{enumerate}
