\chapter{PENUTUP}
\label{chap:penutup}

% Ubah bagian-bagian berikut dengan isi dari penutup

\section{Kesimpulan}
\label{sec:kesimpulan}

Berdasarkan hasil penelitian yang telah dilakukan dapat ditarik kesimpulan-kesimpulan berikut:

\begin{enumerate}[nolistsep]
	
\item Metode CNN dengan arsitektur Residual Neural Network dengan menerapkan transfer learning dapat digunakan untuk mengklasifikasikan penyakit retinopati diabetik
\item Dari masing-masing skenario, didapatkan tiga buah model yaitu best validated model dan best trained model atau model dengan validasi terbaik dan akurasi training terbaik dari 100 epoch, dan last model atau model dari epoch ke 100. Antara kedua model tersebut didapatkan bahwa best validated model menghasilkan performa lebih baik dibanding last model.
\item Tanpa menggunakan penyesuaian beban pada class, akurasi tertinggi berhasil didapatkan oleh ResNet 18 dengan akurasi 82,1%
\item Menggunakan penyesuaian beban pada class, akurasi tertinggi berhasil didapatkan oleh ResNet 18 dengan akurasi 81,3%
\item Pada Quadratic Weighted Kappa, best validated model memiliki nilai yang cenderung rendah disbanding dengan last model ataupun best trained model. 
\item Nilai tertinggi Quadratic weighted kappa pada model tanpa penyesuaian beban adalah resnet18 dengan nilai 0,758
\item Nilai tertinggi Quadratic weighted kappa pada model dengan penyesuaian beban adalah resnet18 dengan nilai 0,752
\item Pada best trained model, penambahan beban pada class menyebabkan perkembangan pada nilai Quadratic Weighted Kappa terkecuali pada resnet18 dan resnet 101, yang mana keduanya mengalami penurunan sebesar 0,006 dan 0,04
\item Pada best validated model, penambahan beban pada class justru menurunkan nilai Quadratic Weighted Kappa, kecuali pada resnet 50 dan resnet 101.
\item Pada model terakhir dari 100 epoch, penambahan beban pada class mengakibatkan penurunan pada nilai Quadratic Weighted Kappa kecuali pada resnet 152


\end{enumerate}

\section{Saran}
\label{chap:saran}

Berikut adalah beberapa saran yang dapat diberikan pada penelitian untuk tugas akhir ini:

\begin{enumerate}[nolistsep]
	
\item 

\end{enumerate}
