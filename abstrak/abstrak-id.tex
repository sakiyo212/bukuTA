\begin{center}
  \large\textbf{ABSTRAK}
\end{center}

\addcontentsline{toc}{chapter}{ABSTRAK}

\vspace{2ex}

\begingroup
% Menghilangkan padding
\setlength{\tabcolsep}{0pt}

\noindent
\begin{tabularx}{\textwidth}{l >{\centering}m{2em} X}
  Nama Mahasiswa    & : & \name{}         \\

  Judul Tugas Akhir & : & \tatitle{}      \\

  Pembimbing        & : & 1. \advisor{}   \\
                    &   & 2. \coadvisor{} \\
\end{tabularx}
\endgroup

% Ubah paragraf berikut dengan abstrak dari tugas akhir
\emph{Retinopati diabetik (DR) adalah komplikasi mikrovaskular diabetes dan merupakan penyebab utama kebutaan di antara orang dewasa usia kerja di seluruh dunia. Deteksi dan intervensi dini sangat penting untuk mencegah kehilangan penglihatan dan meningkatkan hasil pengobatan pasien. Namun, metode skrining tradisional sering kali memiliki keterbatasan dalam hal akurasi dan aksesibilitas. Penelitian ini mengusulkan penerapan Residual Neural Network (ResNet) untuk deteksi dan klasifikasi DR secara otomatis dari gambar fundus. Penelitian ini bertujuan untuk berkontribusi pada kemajuan diagnosis DR otomatis dan pada akhirnya meningkatkan perawatan pasien melalui intervensi dini dan strategi perawatan yang dipersonalisasi.}

% Ubah kata-kata berikut dengan kata kunci dari tugas akhir
\textbf{Kata Kunci: \emph{Retinopati Diabetik, ResNet, Deep Learning, Analisis Angiografi OCT}}
