\begin{center}
  \large\textbf{ABSTRACT}
\end{center}

\addcontentsline{toc}{chapter}{ABSTRACT}

\vspace{2ex}

\begingroup
% Menghilangkan padding
\setlength{\tabcolsep}{0pt}

\noindent
\begin{tabularx}{\textwidth}{l >{\centering}m{3em} X}
  \emph{Name}     & : & \name{}         \\

  \emph{Title}    & : & \engtatitle{}   \\

  \emph{Advisors} & : & 1. \advisor{}   \\
                  &   & 2. \coadvisor{} \\
\end{tabularx}
\endgroup

% Ubah paragraf berikut dengan abstrak dari tugas akhir dalam Bahasa Inggris
\emph{Diabetic retinopathy (DR) is a microvascular complication of diabetes and is the leading cause of blindness among working-age adults worldwide. Early detection and intervention are crucial to prevent vision loss and improve patient outcomes. However, traditional screening methods often face limitations in accuracy and accessibility. This study proposes the implementation of a Residual Neural Network (ResNet) for automated DR detection and classification from fundus images. By achieving these objectives, this study aims to contribute to the advancement of automated DR diagnosis and ultimately improve patient care through early intervention and personalized treatment strategies.}

% Ubah kata-kata berikut dengan kata kunci dari tugas akhir dalam Bahasa Inggris
\textbf{Keywords: \emph{Diabetic retinopathy, ResNet, Deep Learning, Optical Coherence Tomography Angiography Analysis}}