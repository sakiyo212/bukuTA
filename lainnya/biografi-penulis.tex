\begin{center}
  \Large
  \textbf{BIOGRAFI PENULIS}
\end{center}

\addcontentsline{toc}{chapter}{BIOGRAFI PENULIS}

\vspace{2ex}

\begin{wrapfigure}{L}{0.3\textwidth}
  \centering
  \vspace{-3ex}
  % Ubah file gambar berikut dengan file foto dari mahasiswa
  \includegraphics[width=0.3\textwidth]{gambar/foto diri.png}
  \vspace{-4ex}
\end{wrapfigure}

% Ubah kalimat berikut dengan biografi dari mahasiswa
\name{}, lahir di kota kecil bernama Pati pada tanggal 2 Januari 2002. Lahir sebagai anak pertama dari dua bersaudara. Setelah lulus dari SMA Negeri 1 Pati, penulis melanjutkan pendidikan ke Perguruan Tinggi Institut Teknologi Sepuluh Nopember, program studi S1 Teknik Komputer, Fakultas Teknologi Elektro \& Informatika Cerdas (FTEIC).

Selama berkuliah, penulis mengikuti kegiatan akademis yaitu Bangkit 2022 dengan \emph{path} pembelajaran komputasi awan. Penulis juga mengikuti kegiatan ekstrakurikuler yaitu staff ukm IFLS, staff UKM IAC, dan menjadi anggota dari Legislatif Mahasiswa Departemen Teknik Komputer.

Pada Penelitian tugas akhir, penulis awalnya memilih untuk mengangkat topik pembuatan gim berdasarkan sejarah. Namun dikarenakan kendala, topik tersebut tidak dapat dilanjutkan, dan penulis memilih untuk melakukan penelitian di bidan pembelajaran mesin.

%Satrio memulai pendidikannya di TK PGRI Sumbermulyo, kemudian melanjutkan pendidikan di SD Sumbermulyo 03. Pada kelas 4, SD ini memiliki murid terlalu sedikit sehingga harus diberhentikan operasionalnya. Satrio memutuskan untuk pindah ke SD Sumbermulyo 02, yang mana lebih dekat jaraknya ke rumahnya. 
%
%Setelah lulus dari SD, satrio melanjutkan pendidikannya ke SMP Negeri 1 Pati. Hal ini berbeda dengan kebanyakan teman sekelasnya yang memilih untuk melanjutkan ke SMP di daerah winong ataupun Gabus. Satrio menjadi satu-satunya murid dari desanya yang bersekolah di SMPN 1 Pati kala itu.
%
%Satrio lulus dari SMP N 1 Pati dengan Nilai UN sebesar 36,05 dan melanjutkan pendidikannya ke SMA N 1 Pati. Di sana satrio mengikuti ekstrakurikuler paskibra dan sinematografi. Sempat mengikuti OSN pada tahun 2018, namun hanya berhasil sampai ke tingkat provinsi.