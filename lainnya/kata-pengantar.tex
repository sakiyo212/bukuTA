\begin{center}
  \Large
  \textbf{KATA PENGANTAR}
\end{center}

\addcontentsline{toc}{chapter}{KATA PENGANTAR}

\vspace{2ex}

% Ubah paragraf-paragraf berikut dengan isi dari kata pengantar

Puji dan syukur penulis ucapkan kepada Tuhan Yang Maha Esa karena berkat rahmat-Nya dapat diselesaikan penelitian serta penyusunan tugas akhir ini yang berjudul "Analisis Retinopati Diabetik dengan Implementasi Menggunakan \emph{Residual Neural Network}."

Penelitian ini disusun dalam rangka pemenuhan bidang riset di Departemen Teknik Komputer dan persyaratan untuk menyelesaikan pendidikan S1. Dalam pelaksanaan penelitian serta penyusunan buku ini, penulis ingin mengucapkan terima kasih kepada:

\begin{enumerate}[nolistsep]

  \item Keluarga yang telah mendukung penulis hingga mencapai titik ini.

  \item Bapak Supeno Mardi Susiki Nugroho, S.T., M.T. dan Bapak Reza Fuad Rachmadi, S.T., M.T., Ph. D selaku pembimbing yang telah membantu mengarahkan penulis selama pengerjaan dan penyusunan buku ini. 

  \item Bapak Dr. Supeno Mardi Susiki Nugroho, S.T., M.T. selaku Kepala Departemen Teknik Komputer, Fakultas Teknologi Elektro dan Informatika Cerdas, Institut Teknologi Sepuluh Nopember Surabaya, yang telah memberikan kesempatan bagi penulis untuk melakukan serta menyelesaikan tugas akhir ini.
  
  \item Bapak dan Ibu dosen di Departemen Teknik Komputer, yang telah memberikan ilmu bagi penulis sehingga dapat dipahami berbagai macam bidang yang diperlukan untuk melaksanakan dan menyelesaikan tugas akhir ini.

  \item Seluruh sahabat dan teman-teman mahasiswa dan staf di ITS, terutama teman-teman Teknik Komputer yang telah membantu penulis dalam berbagai tahap pada pelaksanaan serta penyusunan tugas akhir ini.

\end{enumerate}

Akhir kata, Penulis harap penelitian ini dapat bermanfaat dan berguna untuk sebanyak mungkin orang. Penulis sadar bahwa buku ini masih belum sempurna oleh karena itu diharapkan dapat diberikan kritik dan saran sehingga penulis dapat lebih meningkatkan kualitas keluaran penulis untuk kedepannya.

\begin{flushright}
  \begin{tabular}[b]{c}
    \place{}, \MONTH{} \the\year{} \\
    \\
    \\
    \\
    \\
    \name{}
  \end{tabular}
\end{flushright}
