\chapter*{ABSTRAK}
\begin{center}
  \large
  \textbf{\tatitle{}}
\end{center}
\addcontentsline{toc}{chapter}{ABSTRAK}
% Menyembunyikan nomor halaman
\thispagestyle{empty}

\begin{flushleft}
  \setlength{\tabcolsep}{0pt}
  \bfseries
  \begin{tabular}{ll@{\hspace{6pt}}l}
    Nama Mahasiswa / NRP & : & \name{} / \nrp{}                      \\
    Departemen           & : & \studyprogram{} \facultyshort{} - ITS \\
    Dosen Pembimbing     & : & 1. \advisor{}                         \\
                         &   & 2. \coadvisor{}                       \\
  \end{tabular}
  \vspace{4ex}
\end{flushleft}
\textbf{Abstrak}

% Isi Abstrak
Retinopati diabetik (DR) adalah komplikasi mikrovaskular diabetes dan merupakan penyebab utama kebutaan di antara orang dewasa usia kerja di seluruh dunia. Deteksi dan intervensi dini sangat penting untuk mencegah kehilangan penglihatan dan meningkatkan hasil pengobatan pasien. Namun, metode skrining tradisional sering kali memiliki keterbatasan dalam hal akurasi dan aksesibilitas. Penelitian ini mengusulkan penerapan Residual Neural Network (ResNet) untuk deteksi dan klasifikasi DR secara otomatis dari gambar fundus. Penelitian ini bertujuan untuk berkontribusi pada kemajuan diagnosis DR otomatis dan pada akhirnya meningkatkan perawatan pasien melalui intervensi dini dan strategi perawatan yang dipersonalisasi.

\vspace{2ex}
\noindent
\textbf{Kata Kunci: \emph{Retinopati Diabetik, ResNet, Deep Learning, Analisis Citra Fundus}}