% Atur variabel berikut sesuai namanya

% nama
\newcommand{\name}{Satrio Heru Utomo}
\newcommand{\authorname}{Utomo, Satrio Heru}
\newcommand{\nickname}{Satrio}
\newcommand{\advisor}{Dr. Supeno Mardi Susiki Nugroho, S.T., M.T.}
\newcommand{\coadvisor}{Reza Fuad Rachmadi, S.T., M.T., Ph. D}
\newcommand{\examinerone}{Dr. Surya Sumpeno , S.T., M.Sc.}
\newcommand{\examinertwo}{Dr. Susi Juniastuti, S.T., M.Eng.}
\newcommand{\examinerthree}{Arta Kusuma Hernanda, S.T., M.T}
\newcommand{\headofdepartment}{Dr. Supeno Mardi Susiki Nugroho, S.T., M.T.}

% identitas
\newcommand{\nrp}{0721 19 4000 0053}
\newcommand{\advisornip}{19700313 199512 1 001}
\newcommand{\coadvisornip}{19850403 201212 1 001}
\newcommand{\examineronenip}{19690613 199702 1}
\newcommand{\examinertwonip}{19650618 199903 2}
\newcommand{\examinerthreenip}{19962023 11024}
\newcommand{\headofdepartmentnip}{19700313 199512 1 001}

% judul
\newcommand{\tatitle}{ANALISIS RETINOPATI DIABETIK DENGAN IMPLEMENTASI MENGGUNAKAN \emph{RESIDUAL NEURAL NETWORK}}
\newcommand{\engtatitle}{ANALYSIS OF DIABETIC RETINOPATHY WITH IMPLEMENTATION USING \emph{RESIDUAL NEURAL NETWORK}}

% tempat
\newcommand{\place}{Surabaya}

% jurusan
\newcommand{\studyprogram}{Teknik Komputer}
\newcommand{\engstudyprogram}{Computer Engineering}

% fakultas
\newcommand{\faculty}{Teknologi Elektro dan Informatika Cerdas}
\newcommand{\engfaculty}{Intelligent Electrical and Informatics Technology}

% singkatan fakultas
\newcommand{\facultyshort}{FTEIC}
\newcommand{\engfacultyshort}{ELECTICS}

% departemen
\newcommand{\department}{Teknik Komputer}
\newcommand{\engdepartment}{Computer Engineering}

% kode mata kuliah
\newcommand{\coursecode}{EC234801}
